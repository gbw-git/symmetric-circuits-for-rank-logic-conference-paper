
\documentclass[a4paper,UKenglish]{lipics-v2018}
% This is a template for producing LIPIcs articles. See lipics-manual.pdf for
% further information. for A4 paper format use option "a4paper", for US-letter
% use option "letterpaper" for british hyphenation rules use option "UKenglish",
% for american hyphenation rules use option "USenglish" for section-numbered
% lemmas etc., use "numberwithinsect"

%\usepackage{microtype}%if unwanted, comment out or use option "draft"


\usepackage{xspace}
\usepackage{mymacros}

% 
\graphicspath{{./graphics/}}%helpful if your graphic files are in another directory

\bibliographystyle{plainurl}% the recommnded bibstyle

\title{Symmetric Circuits for Rank Logic}

% \titlerunning{Dummy short
% title}%optional, please use if title is longer than one line

\author{Anuj Dawar}{University of Cambridge Computer Laboratory\\{Cambridge,
    UK}}{anuj.dawar@cl.cam.ac.uk}{}{}%mandatory, please use full name; only 1 author per \author macro; first two parameters are mandatory, other parameters can be empty.

\author{Gregory Wilsenach}{University of Cambridge Computer
  Laboratory\\{Cambridge, UK}}{gregory.wilsenach@cl.cam.ac.uk}{}{Funding
  provided by the Gates Cambridge Scholarship}

\authorrunning{J.Q. Public and J.R.
  Public}%mandatory. First: Use abbreviated first/middle names. Second (only in severe cases): Use first author plus 'et. al.'

\Copyright{Anuj Dawar and Gregory
  Wilsenach} %mandatory, please use full first names. LIPIcs license is "CC-BY"; http://creativecommons.org/licenses/by/3.0/

\subjclass{F.1.3 Complexity Measures and Classes, F.4.1 Math.
  Logic} % mandatory: Please choose ACM 2012 classifications from https://www.acm.org/publications/class-2012 or https://dl.acm.org/ccs/ccs_flat.cfm . E.g., cite as "General and reference $\rightarrow$ General literature" or \ccsdesc[100]{General and reference~General literature}.

\keywords{Dummy keyword}%mandatory

\category{}%optional, e.g. invited paper

\relatedversion{}%optional, e.g. full version hosted on arXiv, HAL, or other respository/website

\supplement{}%optional, e.g. related research data, source code, ... hosted on a repository like zenodo, figshare, GitHub, ...

\funding{}%optional, to capture a funding statement, which applies to all authors. Please enter author specific funding statements as fifth argument of the \author macro.

\acknowledgements{I want to thank \dots}%optional

% Editor-only macros:: begin (do not touch as
% author)%%%%%%%%%%%%%%%%%%%%%%%%%%%%%%%%%%
\EventEditors{John Q. Open and Joan R. Access} \EventNoEds{2}
\EventLongTitle{42nd Conference on Very Important Topics (CVIT 2016)}
\EventShortTitle{CVIT 2016} \EventAcronym{CVIT} \EventYear{2016}
\EventDate{December 24--27, 2016} \EventLocation{Little Whinging, United
  Kingdom} \EventLogo{} \SeriesVolume{42} \ArticleNo{23}
% \nolinenumbers %uncomment to disable line numbering
% \hideLIPIcs %uncomment to remove references to LIPIcs series (logo, DOI, ...), e.g. when preparing a pre-final version to be uploaded to arXiv or another public repository
%%%%%%%%%%%%%%%%%%%%%%%%%%%%%%%%%%%%%%%%%%%%%%%%%%%%%%

\begin{document}

\maketitle

\begin{abstract}
We give a circuit characterization for fixed-point logic with rank
(FPR) in terms of families of symmetric circuits with rank gates,
along the lines of that for fixed-point logic with counting (FPC)
given by [Anderson and Dawar 2017].  This requires the development of
a broad framework of circuits in which the individual gates compute
functions that are not symmetric (i.e., invariant under all
permutations of their inputs).  In the case of FPC, the proof of
equivalence of circuits and logic rests heavily on the assumption that
individual gates compute such symmetric function and so novel
techniques are required to make this work for FPR.
\end{abstract}

\section{Introduction}

The study of extensions of fixed-point logics plays an important role in the
field of descriptive complexity theory. In particular, fixed-point logic with
counting ($\FPC$) has become a reference logic in the search for a logic for
polynomial-time (see~\cite{Dawar-siglog}). In this context, Anderson and
Dawar~\cite{AndersonD17} provide an interesting characterisation of the
expressive power of $\FPC$ in terms of circuit complexity. They show that the
properties expressible in this logic are exactly those that can be decided by
polynomially-uniform families of circuits (with threshold gates) satisfying a
natural \emph{symmetry} condition. Not only does this illustrate the robustness
of $\FPC$ as a complexity class within $\PT$ by giving a distinct and natural
characterization of it, it also demonstrates that the techniques for proving
inexpressibility in the field of finite model theory can be understood as
lower-bound methods against a natural circuit complexity class. This raises an
obvious question (explicitly posed in the concluding section
of~\cite{AndersonD17}) of how to obtain circuit characterizations of logics more
expressive than $\FPC$, such as choiceless polynomial time (CPT) and fixed-point
logic with rank ($\FPR$). It is this last question that we address in this
paper.

Fixed-point logic with rank extends the expressive power of $\FPC$ by means of
operators that allow us to define the rank of a matrix over a finite field. Such
operators are natural extensions of counting---counting the dimension of a
definable vector space rather than just the size of a definable set. At the same
time they make the logic rich enough to express many of the known examples that
separate $\FPC$ from $\PT$. Rank logics were first introduced
in~\cite{Dawar09logicswith}. The version $\FPR$ we consider here is that defined
by Gr\"adel and Pakusa~\cite{GradelP15a} where the prime characteristic is a
parameter to the rank operator, and we do not have a distinct operator for each
prime number. Formal definitions of these logics are given in
Section~\ref{sec:background}. We give a circuit characterization, in terms of
symmetric circuits, of $\FPR$. One might think, at first sight, that this is a
simple matter of extending the circuit model with gates for computing the rank
of a matrix. It turns out, however, that the matter is not so simple as the
symmetry requirement interacts in surprising ways with such rank gates. It
requires a new framework for defining classes of such circuits, which yields
remarkable new insights.

It is important to clarify at the outset what is meant by symmetric circuits.
Indeed, there are different senses in which the word \emph{symmetry} is used in
the context of circuits (and also in this paper). We say that a Boolean function
$f:\{0,1\}^n \ra \{0,1\}$ is symmetric if the value of the function on a string
$s$ is determined by the number of $1$s in $s$. In other words, $f$ is invariant
under \emph{all} permutations of its input. In contrast, when we consider the
input to a Boolean function to be the adjacency matrix of an $n$-vertex graph,
for example, and $f : \{0,1\}^{n \choose 2} \ra \{0,1\}$ decides a graph
property, then $f$ is invariant under all permutations of its input induced by
permutations of the $n$ vertices of the graph. We call such a function
\emph{graph-invariant}. More generally, for a relational vocabulary $\tau$ and a
standard encoding of $n$-element $\tau$-structures as strings over $\{0,1\}$, we
can say that function taking such strings as input is $\tau$-invariant if it is
invariant under permutations induced by the $n$ elements. A circuit $C$
computing such an invariant function is said to be \emph{symmetric} if every
permutation of the $n$ elements extends to an automorphism of $C$. It is
families of symmetric circuits in this sense that characterize $\FPC$
in~\cite{AndersonD17}. The restriction to symmetric circuits arises naturally in
the study of logics and has appeared previously under the names of generic
circuits in the work of~\cite{DENENBERG1986216} and explicitly order-invariant
circuits in the work of Otto~\cite{Otto1997}.

The main result of~\cite{AndersonD17} says that the properties of
$\tau$-structures definable in $\FPC$ are exactly those that can be decided by
$\PT$-uniform families of symmetric circuits using AND, OR, NOT and majority
gates. Note that each of these gates itself computes a Boolean function that is
symmetric in the strong sense identified above. On the other hand, a gate for
computing a rank threshold function, e.g.\ one that takes as input a $n \times
n$ matrix and outputs $1$ if the rank of the matrix is greater than a threshold
$t$, is \emph{not} symmetric. In our circuit characterization of $\FPR$ we
necessarily have to consider such non-symmetric gates. Indeed, we show in
Section~\ref{sec:symm-circ} that $\PT$-uniform families of symmetric circuits
using gates for \emph{any} symmetric functions do not take us beyond the power
of $\FPC$. This is a further illustration of the robustness of $\FPC$. In order
to go beyond it, we need to introduce gates for Boolean functions that are not
symmetric. We construct a systematic framework for including functions computing
$\tau$-invariant functions for arbitrary multi-sorted relational vocabularies
$\tau$ in Section~\ref{sec:symm-circ}. We also explore what it means for such
circuits to be symmetric.

The proof of the circuit characterization of $\FPC$ relies on the \emph{support
  theorem} proved in~\cite{AndersonD17}. This establishes that for any
$\PT$-uniform family of circuits using AND, OR, NOT and majority gates there is
a constant $k$ such that every gate has a support of size at most $k$. That is
to say that we can associate with every gate $g$ in the circuit $C_n$ (the
circuit in the family that works on $n$-element structures) a subset $X$ of
$[n]$ of size at most $k$ such that any permutation of $[n]$ fixing $X$
pointwise extends to an automorphism of $C_n$ that fixes $g$. This theorem is
crucial to the translation of the family of circuits into a formula of $\FPC$,
which is the difficult (and novel) direction of the equivalence. In attempting
to do the same with circuits that now use rank-threshold gates we are faced with
the difficulty that the proof of the support theorem in~\cite{AndersonD17}
relies in an essential way on the fact that the Boolean function computed at
each gate is symmetric. We are able to overcome this difficulty and prove a
support theorem for circuits with rank gates but this requires substantial,
novel technical machinery, which we develop in Section~\ref{sec:symm-support}.

Another crucial ingredient in the proof of Anderson and Dawar is that we can
eliminate redundancy in the circuit $C_n$ by making it \emph{rigid}. That is, we
can ensure that the \emph{only} automorphisms of $C_n$ are those that are
induced by permutations of $[n]$. Here we face the difficulty that identifying
the symmetries and eliminting redundancy in a circuit that involves gates
computing $\tau$-invariant functions requires us to solve the isomorphism
problem for $\tau$-structures. This is a hard problem (or, at least, one that we
do not know how to solve efficiently) even when the $\tau$-structures are
$0$-$1$-matrices. We overcome this difficulty by placing a further restriction
on circuits that we call \emph{transparency}. Circuits satisfying this condition
have the property that their lack of redundancy is transparent. We explore this
condition and its consequences in Section~\ref{sec:transparent}, where we also
demonstrate that something like this requrement is necessary. Natural weakenings
of this requirement yield circuits in which deciding symmetry seems hard.

In the characterization of $\FPC$, the translation from formulas into families
of circuits is easy and, indeed, standard. In our case, we have to show that
formulas of $\FPR$ translate into uniform families of circuits using
rank-threshold gates that are symmetric and transparent. This is somewhat more
involved technically and the details are presented in
Section~\ref{sec:formulas-to-circuits}. Finally, with all these tools in place,
the translation of such $\PT$-uniform families of circuits into formulas of
$\FPR$ given in Section~\ref{sec:circuits-to-formulas} completes the
characterization. This still requires substantial new techniques. The
translation of circuits to formulas in~\cite{AndersonD17} relies on the fact
that in order to evaluate a gate computing a symmetric Boolean function, it
suffices to count the number of inputs that evaluate to true and there is a
bijection between the orbits of a gate and tuple assignments to its support.
When counting is no longer sufficient, this bijection has to preserve more
structure and demonstrating this in the case of matrices requires new insight.


\section{Background}
We assume the reader is familiar with first-order logic ($\FO$), inflationary
fixed-point logic ($\FP$), Fixed-point logic with counting ($\FPC$), and
first-order logic with counting quantifiers ($\FOC$). For details on these
logics please see \cite{grohe2017descriptive, immerman1999descriptive}.
Throughout this paper we assume all algebraic structures are finite.

\subsection{Logic}
\subsection{Rank Logic}
\subsection{Circuits}

\section{Generalising Symmetric Circuits}
A Boolean circuit $C$ is usually taken to be a directed acyclic graph with a set
of input gates labelled by variables and with each internal gate $g$ in the
circuit labelled with a function $f_g$ from a basis $\BB$. However, if $f_g$
were allowed to be arbitrary, then an order would need to be imposed on the
children of $g$ in order to ensure unambiguous evaluation. As such, when we say
that $C$ is a directed acyclic graph, and we include no structure on the
children of each gate, we implicitly assume that $f_g$ is invariant under all
permutations of its inputs -- i.e.\ $f_g$ is a symmetric function. The, often
unstated assumption that the basis over which the circuit is defined contain
only symmetric functions is pervasive in circuit complexity. Indeed, it is easy
to see that the standard Boolean basis consisting of $\AND$, $\OR$ and $\NOT$,
as well as bases with majority or threshold functions only contain symmetric
functions.

Anderson and Dawar~\cite{AndersonD17} have defined a characterisation of $\FPC$


\section{Symmetry and Supports}
\section{Transparent Circuits}
\section{The Translation from Formulas into Circuits}
\section{The Translation from Circuits into Formulas}
\section{Concluding Remarks and Future Work}

% \appendix

\bibliography{references.bib}
%\bibliography{lipics-v2018-sample-article}


\end{document}
